\usepackage{glossaries}

\makeglossaries{}

\newglossaryentry{CI/CD}{
  name={CI/CD},
  description={Continuous Integration and Continuous Development. The process of
  automating software testing, building, and deployment.}
}

\newglossaryentry{CLI}
{
    name={CLI},
    description={A command-line interface (CLI) is a text-based user interface (UI) used to run programs, manage computer files and interact with the computer.}
}

\newglossaryentry{DSL}
{
    name=DSL,
    description={A computer language that is specialized to a particular application/domain}
}

\newglossaryentry{target execution}
{
    name=target execution,
    description={An instance of a \gls{target} that needs to be completed as per described in the project definition}
}

\newglossaryentry{cache}
{
    name=cache,
    description={to save something to computers memory or local storage. An example would be the output of a \gls{target execution}}
}

\newglossaryentry{project definition}
{
    name=project definition,
    description={A file containing information pertaining a project, an example could be the language that was used}
}

\newglossaryentry{metadata}
{
    name=metadata,
    description={any custom data that the project owner wants to add about the project}
}

\newglossaryentry{target}
{
    name=target,
    description={An execution target, a recipe or command for what you want to do to/with the project, ie build and deploy the project, run tests, ect}
}

\newglossaryentry{monorepo}
{
  name=monorepo,
  description={A software-development strategy in which the code for a number of projects is stored in the same repository.}
}
